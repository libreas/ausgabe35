Gegenwärtig erlebt der Rückbezug auf vermeintlich bedrohte nationale und
kulturelle Identitäten weltweit einen Aufschwung. Ursachen dafür werden
oft in den, durch zahlreiche Kriegs- und Konfliktherde ausgelösten,
Bewegungen von Flüchtlingen ausgemacht, die es in das Zentrum
kapitalistischer Wohlfahrtsstaaten zieht. Die Instrumentalisierung von
Abstiegsängsten bei den Mittel- und Unterschichten betroffener Länder,
wird aber nicht nur auf den digitalen Plattformen sozialer Medien
betrieben, sondern auch auf den etablierten Podien und in den
traditionellen Institutionen öffentlicher Wissensvermittlung. Die
rechtsnationalen bis rechtsradikalen Stimmen inszenieren sich
gegenwärtig oft als Opfer, weil sie angeblich nicht mehr gehört würden.
Sie empören sich in diesem Zusammenhang, weil Sie eine angebliche
Mehrheitsmeinung aussprechen, die vermeintlich lange genug durch
liberale Wortkosmetik tabuisiert wurde. Durch immer professionellere
Medienstrategien haben sie aber längst erreicht, dass ihre Stimmen im
Gegenteil unverhältnismäßig oft gehört und medial verbreitet werden. Zu
den Bühnen, die sie wählen, gehören Zeitungen, soziale Netzwerke,
Talkshows, Podiumsdiskussionen und Demonstrationen ebenso, wie etwa die
Bestände von Buchhandlungen und Bibliotheken. Im Spannungsfeld von
Neutralität, Informationsfreiheit und den Schutzrechten diskriminierter
Minderheiten, stellt sich auch dem Fachpersonal öffentlicher
Bibliotheken die Frage, wie sie die Öffentlichkeit darin unterstützen
kann, die Debatten nicht von Rechtsnationalen und Rechtsextremen
vereinnahmen zu lassen. Ohne Zensur auszuüben oder für bestimmte Gruppen
Partei zu ergreifen, hat das Fachpersonal von öffentlichen Bibliotheken
die schwierige Herausforderung zu bewältigen, darauf eine Antwort zu
finden.
