\documentclass[a4paper,
fontsize=11pt,
%headings=small,
oneside,
numbers=noperiodatend,
parskip=half-,
bibliography=totoc,
final
]{scrartcl}

\usepackage{synttree}
\usepackage{graphicx}
\setkeys{Gin}{width=.4\textwidth} %default pics size

\graphicspath{{./plots/}}
\usepackage[ngerman]{babel}
\usepackage[T1]{fontenc}
%\usepackage{amsmath}
\usepackage[utf8x]{inputenc}
\usepackage [hyphens]{url}
\usepackage{booktabs} 
\usepackage[left=2.4cm,right=2.4cm,top=2.3cm,bottom=2cm,includeheadfoot]{geometry}
\usepackage{eurosym}
\usepackage{multirow}
\usepackage[ngerman]{varioref}
\setcapindent{1em}
\renewcommand{\labelitemi}{--}
\usepackage{paralist}
\usepackage{pdfpages}
\usepackage{lscape}
\usepackage{float}
\usepackage{acronym}
\usepackage{eurosym}
\usepackage[babel]{csquotes}
\usepackage{longtable,lscape}
\usepackage{mathpazo}
\usepackage[normalem]{ulem} %emphasize weiterhin kursiv
\usepackage[flushmargin,ragged]{footmisc} % left align footnote
\usepackage{ccicons} 
\setcapindent{0pt} % no indentation in captions

%%%% fancy LIBREAS URL color 
\usepackage{xcolor}
\definecolor{libreas}{RGB}{112,0,0}

\usepackage{listings}

\urlstyle{same}  % don't use monospace font for urls

\usepackage[fleqn]{amsmath}

%adjust fontsize for part

\usepackage{sectsty}
\partfont{\large}

%Das BibTeX-Zeichen mit \BibTeX setzen:
\def\symbol#1{\char #1\relax}
\def\bsl{{\tt\symbol{'134}}}
\def\BibTeX{{\rm B\kern-.05em{\sc i\kern-.025em b}\kern-.08em
    T\kern-.1667em\lower.7ex\hbox{E}\kern-.125emX}}

\usepackage{fancyhdr}
\fancyhf{}
\pagestyle{fancyplain}
\fancyhead[R]{\thepage}

% make sure bookmarks are created eventough sections are not numbered!
% uncommend if sections are numbered (bookmarks created by default)
\makeatletter
\renewcommand\@seccntformat[1]{}
\makeatother

% typo setup
\clubpenalty = 10000
\widowpenalty = 10000
\displaywidowpenalty = 10000

\usepackage{hyperxmp}
\usepackage[colorlinks, linkcolor=black,citecolor=black, urlcolor=libreas,
breaklinks= true,bookmarks=true,bookmarksopen=true]{hyperref}
\usepackage{breakurl}

%meta
%meta

\fancyhead[L]{B. Kaden, M. Kleineberg\\ %author
LIBREAS. Library Ideas, 35 (2019). % journal, issue, volume.
\href{http://nbn-resolving.de/}
{}} % urn 
% recommended use
%\href{http://nbn-resolving.de/}{\color{black}{urn:nbn:de...}}
\fancyhead[R]{\thepage} %page number
\fancyfoot[L] {\ccLogo \ccAttribution\ \href{https://creativecommons.org/licenses/by/4.0/}{\color{black}Creative Commons BY 4.0}}  %licence
\fancyfoot[R] {ISSN: 1860-7950}

\title{\LARGE{Scholarly Makerspaces -- Ein Zwischenbericht zum DFG-Projekt FuReSH}}% title
\author{Ben Kaden, Michael Kleineberg} % author

\setcounter{page}{1}

\hypersetup{%
      pdftitle={Scholarly Makerspaces -- Ein Zwischenbericht zum DFG-Projekt FuReSH},
      pdfauthor={Ben Kaden, Michael Kleineberg},
      pdfcopyright={CC BY 4.0 International},
      pdfsubject={LIBREAS. Library Ideas, 35 (2019).},
      pdfkeywords={FuReSH, Scholarly Makerspace, Konzeptstudie, Universitätsbibliothek, Digital Humanities},
      pdflicenseurl={https://creativecommons.org/licenses/by/4.0/},
      pdfcontacturl={http://libreas.eu},
      baseurl={http://libreas.eu},
      pdflang={de},
      pdfmetalang={de}
     }



\date{}
\begin{document}

\maketitle
\thispagestyle{fancyplain} 

%abstracts

%body
An der Universitätsbibliothek der Humboldt-Universität zu Berlin wird
aktuell eine Konzeptstudie zur Idee der so genannten \emph{Scholarly
Makerspaces} erarbeitet. Mit diesem Ansatz soll die Vermittlung von
Bausteinen der digitalen Forschung in den Kultur- und
Geisteswissenschaften auch über den Kernbereich der Digital Humanities
hinaus auf lokaler Ebene verbessert werden. Aus bibliothekarischer Sicht
versteht sich die Idee zugleich als ein Angebot, um die Rolle von
Universitätsbibliotheken in ihrem jeweiligen Bedingungs- und
Wirkungsrahmen explizit als Partner der Forschung zu untersuchen und
Möglichkeiten einer differenzierten Vermittlung so genannter digitaler
Literarizitäten über den sehr allgemeinen Begriff der
Informationskompetenz hinaus auszuloten. Aus Sicht der Entwicklung von
digitalen Forschungswerkzeugen ermöglichen Scholarly Makerspaces einen
besseren Zugang zu Zielgruppen in der Breite und damit neue
Vermittlungs- und Kommunikationsmöglichkeiten für Anbietende. Der
Bericht dokumentiert den Zwischenstand anhand ausgewählter Aspekte. Die
Gesamtstudie wird Ende des Jahres 2019 vorgelegt.

\hypertarget{projektziel}{%
\section{1 Projektziel}\label{projektziel}}

Das DFG-Projekt \enquote{Future e-Research Support in the Humanities}
(FuReSH)\footnote{Projektwebseite:
  \url{https://www.ub.hu-berlin.de/de/ueber-uns/projekte/furesh}.},
angesiedelt an der Universitätsbibliothek der Humboldt-Universität zu
Berlin, erarbeitet eine Konzeptstudie zur Implementierbarkeit von
sogenannten \enquote{Scholarly Makerspaces} in das
Dienstleistungsangebot von Universitätsbibliotheken. Den Ausgangspunkt
bildet die Annahme, dass das traditionelle, primär auf bestandsbezogene
Bereitstellung und Erschließung fokussierende Serviceverständnis
Wissenschaftlicher Bibliotheken nicht ausreicht, um datenintensive und
toolbasierte digitale Arbeits- und Forschungsformen organisatorisch,
fachlich und technisch in der notwendigen Breite zu unterstützen.

Im Bibliothekswesen greifen digitale Technologien auf Traditionslinien
beispielsweise der Online-Datenbanken, Verbundkatalogisierung oder auch
digitaler Dokumentenlieferung zurück, die seit den 1990er Jahren unter
dem Begriff der \enquote{Digitalen Bibliothek} erheblich erweitert einen
großen Teil des Aufgabenfeldes der Bibliotheken prägen. Selbst wo der
Schwerpunkt auf der Versorgung mit Printmaterialien verbleibt, sind
Nachweis und Retrieval in den überwiegenden Fällen digital. Spätestens
mit der Open-Access-Bewegung und dem Aufkommen von Publikationsservern
und Repositorien werden Wissenschaftliche Bibliotheken und damit auch
Universitätsbibliotheken zu Infrastruktur- und Serviceanbietern für die
digitale Wissenschaft.

Die Konzeption von Scholarly Makerspaces schließt an diese Tradition an
und verknüpft sie mit der Idee der Wissenschaftlichen Bibliothek als
einem \enquote{Labor} der Geisteswissenschaften, in dem Materialien und
Werkzeuge für geisteswissenschaftliche Forschung an einem Ort direkt
vermittelt werden. Hierbei werden digitale Inhalte und Ressourcen nicht
mehr nur nach traditionellem Verständnis als „lokal" vorgehaltene
Bibliotheksbestände gesammelt, erschlossen und für die Nutzung zur
Verfügung gestellt. Vielmehr müssen bestandsübergreifend digitale
Quellenmaterialien beziehungsweise Forschungsdaten für ihre Nutzung und
Nutzbarkeit in vernetzten Forschungskontexten von Bibliotheken
aufbereitet und kuratiert werden.

Zur Unterstützung digitaler geistes- und kulturwissenschaftlicher
Forschung ist damit also nicht nur die Vermittlung digitaler
Forschungsdaten (beispielsweise Text- und Bildmaterialien
beziehungsweise Korpora), sondern auch der entsprechenden Analyse- und
Bearbeitungswerkzeuge, basaler Verständnis- und Nutzungskompetenzen
sowie relevanter übergeordneter Kontexte notwendig. Da entsprechende
Nutzungskompetenzen aufgrund der technischen Anforderungen, der Spezifik
der Lösungen sowie wissenschaftsstruktureller Bedingungen in vielen
Fällen nicht vorausgesetzt werden können, liegt ein Schwerpunkt auf der
Vermittlung einer grundlegenden \emph{digital literacy}, die
ausdrücklich auch die Reflexion über die digitale Transformation in den
Geistes- und Kulturwissenschaften einschließt beziehungsweise auf eine
solche hinleitet.

Der Ansatz der Scholarly Makerspaces ist durch die bislang hauptsächlich
in Öffentlichen Bibliotheken etablierten \enquote{Makerspaces} als
Arbeits- und Experimentierräume inspiriert. In diesen wird eine
ergebnisoffene und Community-orientierte Auseinandersetzung mit
physischen und digitalen Werkzeugen und Technologien in offenen sozialen
Räumen angeregt und ermöglicht. Die Idee der \enquote{Scholarly
Makerspaces} spezifiziert diesen Ansatz dahingehend, dass sie sich
erstens primär auf digitale Werkzeuge bezieht und zweitens dezidiert auf
Forschungskontexte richtet, insbesondere auf die infrastrukturelle
Unterstützung von e-Research in den Geistes- und Kulturwissenschaften.

Zusammengefasst bieten Scholarly Makerspaces offene, dynamische und auf
Rückkopplung orientierte Infrastrukturen für digital geprägte geistes-
und kulturwissenschaftliche Forschung mit den Schwerpunkten:

\begin{itemize}

\item
  Ausrichtung der Serviceinfrastruktur von Universitätsbibliotheken
  sowohl technologisch als auch konzeptionell auf die Bedürfnisse
  vorwiegend der lokalen digitalen Wissenschaftspraxis in den Kultur-
  und Geisteswissenschaften;
\item
  Zugänglichmachung und Kuratierung bestehender digitaler Werkzeuge und
  anderer Ressourcen für die Forschung und Lehre;
\item
  Zugänglichmachung und Kuratierung geeigneter digitaler
  Forschungsmaterialien;
\item
  Kompetenzvermittlung durch Schulungen, Tutorials und weitere
  Veranstaltungen;
\item
  Aufbau eines Kontakt- und Kooperationsnetzwerkes zur Vermittlung von
  Expertise externer Ansprechpartner;
\item
  Stetige Erhebung der Bedarfe und Ansprüche der Fachwissenschaften zur
  künftigen Infra\-struktur- und Serviceentwicklung mit den Zielstellungen
  a) der weiteren iterativen Entwicklung des Angebotes und b) der
  Kommunikation an im Betrachtungsfeld generell aktive Stakeholder;
\item
  Fortlaufendes Monitoring von Trends und neuen Angeboten\footnote{Beispielsweise
    die auch der Logik des Enhanced Publishing folgenden Entwicklung hin
    zu Social Linked Data (Solid), vergleiche
    \url{https://en.wikipedia.org/wiki/Solid_\%28web_decentralization_project\%29}.}
  sowie Prüfung auf ihre jeweilige Relevanz und Integrierbarkeit mit dem
  Serviceprofil der Scholarly Makerspaces.
\end{itemize}

Das Ziel der Konzeptstudie besteht darin, die Realisierbarkeit von
Scholarly Makerspaces unter Einschluss der dafür notwendigen Aufwände
und Bedingungen zu analysieren und die Chancen einer tragfähigen
Umsetzung zu bewerten. Ergebnis der Studie wird ein Prozess- und
Organisationsmodell zur Umsetzung der Scholarly Makerspaces unter
Berücksichtigung von Aufwands- und Kostenkalkulation zur Vorbereitung
einer prototypischen Entwicklung von Scholarly Makerspaces an der
Humboldt-Universität zu Berlin sein.

\hypertarget{verfahren-und-zwischenauswertung}{%
\section{2 Verfahren und
Zwischenauswertung}\label{verfahren-und-zwischenauswertung}}

Um die Projektziele zu erreichen, verfolgt FuReSH mehrere Verfahren zur
Datenerhebung und -auswertung. Als Grundlage einer Bedarfs- und
Anforderungsanalyse werden neben einer Lite\-ra\-tur- und Quellenanalyse
einschließlich einer Sammlung von Best-Practice-Lösungen, Testläufe mit
bestehenden digitalen Werkzeugen vollzogen sowie eine Reihe von
qualitativen Leitfadeninterviews mit Expertinnen und Experten sowohl der
geisteswissenschaftlichen Fach-Commu\-nities als auch des
Infrastrukturbereiches durchgeführt. Zudem stellt das FuReSH-Projekt die
Konzeption von Scholarly Makerspaces auf Veranstaltungen zur Diskussion
und führt Workshops durch, in denen offene Fragen diskutiert werden. Zu
den Kooperationspartnern des Projektes zählen unter anderen die
Forschungsverbünde CLARIN-D\footnote{\url{https://www.clarin-d.net/de/}.}
und DARIAH-DE\footnote{\url{https://de.dariah.eu/}, mittlerweile
  CLARIAH, siehe \url{https://www.clariah.nl/en/}.}, die insbesondere
bei der Auswahl und Vermittlung von digitalen Werkzeugen und Diensten
mit ihrer Expertise zur Verfügung stehen.

\hypertarget{beispiele-digitaler-forschungswerkzeuge}{%
\subsection{2.1 Beispiele digitaler
Forschungswerkzeuge}\label{beispiele-digitaler-forschungswerkzeuge}}

Das Angebot an Ressourcen und Werkzeugen für die digitale Forschung in
den Geistes- und Kulturwissenschaften ist mittlerweile sehr groß und
wird oft als unübersichtlich empfunden.\footnote{Dies bestätigten
  beispielsweise Aussagen aus den FuReSH-Interviews.} Einen generellen
Überblick bieten das nach Forschungsprozessen angeordnete Verzeichnis
\emph{Digital Research Tools}\footnote{\url{https://dirtdirectory.org/}.}
sowie die noch umfangreichere \emph{Database of Scholarly Communication
Tools}\footnote{\url{https://blogs.lse.ac.uk/impactofsocialsciences/2015/11/11/101-innovations-in-scholarly-communication/}.}.
Allgemein ist es sinnvoll zwischen allgemeinen digitalen Werkzeugen (zum
Beispiel Textverarbeitungsprogramme, Literaturverwaltungssysteme,
Cloudspeicher, Etherpads, Wikis, Repositorien) und dezidierten digitalen
Forschungswerkzeugen zu unterscheiden, die zur Analyse von
Forschungsobjekten eingesetzt werden. Allgemeine digitale Werkzeuge
werden in den Scholarly Makerspaces dort für die Vermittlung
berücksichtigt, wo sie einen sinnvollen Mehrwert für die digitale
kultur- und geisteswissenschaftliche Forschung bieten.

Welche konkreten Angebote an digitalen Forschungswerkzeugen in Scholarly
Makerspaces integriert werden, hängt von mehreren Leitkriterien ab:

\begin{itemize}

\item
  den Anforderungen der Zielgruppen,
\item
  der Entwicklungsreife der Werkzeuge,
\item
  der Relevanz und einer eventuellen Etablierung in den Communities,
\item
  den lokalen Vermittlungsressourcen
\item
  und gegebenenfalls die Berücksichtigung und Unterstützung der
  Ansprüche offener Wissenschaft.
\end{itemize}

Zudem sollen die Ansprüche und Bedarfe der jeweils lokal anzutreffenden
Zielgruppen besonders berücksichtigt werden. Nachfolgend werden kurz
einige Beispiele bewährter digitaler Forschungswerkzeuge benannt.

\hypertarget{catma}{%
\subsubsection{CATMA}\label{catma}}

Ein zentrales Anwendungsfeld digitaler Forschung in den Geistes- und
Kulturwissenschaften besteht im Bereich der Annotation. Hierfür steht
mittlerweile eine große Vielfalt von unterschiedlichen Werkzeugen zur
Verfügung, oft Open Source, die je nach Szenario direkt über die
Scholarly Makerspaces angeboten oder vermittelt werden könnten. Ein
Beispiel dafür ist \emph{CATMA} (\emph{Computer Assisted Text Markup and
Analysis}), das webbasierte und kollaborative Funktionen zur
Korpuszusammenstellung, Annotation, Analyse und Visualisierung von
Texten ermöglicht.\footnote{\url{http://catma.de/}.} Die Software
erweist sich hinsichtlich der Funktionalität und Usability als besonders
weit entwickelt und hat zudem niedrige Zugangshürden.

\hypertarget{hypothes.is}{%
\subsubsection{Hypothes.is}\label{hypothes.is}}

Eine weitere niedrigschwellige Anwendung im Bereich der Annotation ist
der so genannte \enquote{Conversation Layer} des
Hypothes.is-Projekts.\footnote{\url{https://web.hypothes.is/}.} Er ist
rein browserbasiert und bildet eine Art Verbindung zu
Social-Media-Anwendungen. Die Lösung ermöglicht es, Komponenten auf
beliebigen Webseiten zu annotieren und über Tags zu erschließen. Sie
bietet sich insbesondere zur Erzeugung eines grundlegenden
Verständnisses des Prinzips der digitalen und vernetzenden Annotation
an. Die Lösung ist dank ihrer Niedrigschwelligkeit insbesondere als
Ausgangspunkt für die methodologische Reflexionsarbeit geeignet.

\hypertarget{voyant-tools}{%
\subsubsection{Voyant Tools}\label{voyant-tools}}

\emph{Voyant Tools} bieten eine niedrigschwellige Arbeitsumgebung, die
digitale Werkzeuge bereitstellt, um Texte quantitativ zu analysieren und
zu visualisieren.\footnote{\url{https://voyant-tools.org/}.} Die
Angebote stehen für eine Nachnutzung zur Verfügung. Sie bieten sich auf
Workstations oder auch als Plattformdienste vor allem zu didaktischen
Zwecken an und ermöglichen nach einer methodologischen Einführung die
Analyse eigener Korpora.

\hypertarget{stereoscope}{%
\subsubsection{Stereoscope}\label{stereoscope}}

\emph{Stereoscope} ist eine Visualisierungslösung für hermeneutische
Annäherungen an Texte über Annotationen.\footnote{\url{http://stereoscope.threedh.net/}.}
Die Anwendung ist funktional und hinsichtlich der Usability auf einem
für die Anwendung in den Scholarly Makerspaces geeigneten
Entwicklungsstand. Zudem ist sie browserbasiert und offen nutzbar und
bietet Im- und Exportschnittstellen für JSON-Daten.

\hypertarget{dariah-topics-explorer}{%
\subsubsection{DARIAH Topics Explorer}\label{dariah-topics-explorer}}

Ein weiteres wichtiges Anwendungsfeld ist das Topic Modeling, das es
ermöglicht aus großen Textkorpora wichtige Themenfelder zu
identifizieren und zu visualisieren. Das Werkzeug \emph{DA\-RIAH Topics
Explorer} eignet sich für Textdateien in den Formaten TXT oder XML und
kann Ergebnisse in unterschiedlichen Dateiformaten ausgeben.\footnote{\url{https://fortext.net/tools/tools/dariah-topics-explorer}.}

\hypertarget{stylo}{%
\subsubsection{Stylo}\label{stylo}}

Für textorientierte Geistes- und Kulturwissenschaften bieten
stylometrische Analyseverfahren ein weiteres Anwendungsfeld. Das
Werkzeug \emph{Stylo} ermöglicht durch stilistische Vergleichsanalysen
Fragen der Autorschaftsattribution, Genre- oder Epochenklassifikationen
sowie stilistische Entwicklungen eines Gesamtwerkes zu
bearbeiten.\footnote{\url{https://fortext.net/tools/tools/stylo}.}

\hypertarget{transkribus}{%
\subsubsection{Transkribus}\label{transkribus}}

Um handschriftliche Quellen digital nutzbar zu machen, müssen diese
digitalisiert und aufbereitet werden. Das Werkzeug \emph{Transkribus}
wurde für das manuelle Transkribieren und die automatisierte
Handschriftenerkennung (HTR) entwickelt. Zudem wird eine optische
Zeichenerkennung (OCR) für Druckschriften angeboten. Das Werkzeug eignet
sich auch zur Erstellung digitaler Editionen.\footnote{\url{https://fortext.net/tools/tools/transkribus}.}

\hypertarget{conedakor}{%
\subsubsection{ConedaKOR}\label{conedakor}}

Das webbasierte Werkzeug \emph{ConedaKOR} dient der Verwaltung und
Präsentation akademischer Objektsammlungen aus den bildbasierten Kultur-
und Geisteswissenschaften.\footnote{\url{https://coneda.net/kor/}.} Es
beruht auf der Technologie einer Graphdatenbank, die semantische
Informationen in einer Netzstruktur verknüpft und für eine interaktive
Recherche nutzbar macht.

In den Testläufen mit einer Reihe von Digital-Humanities-Tools wurde
jedoch auch deutlich, dass es insgesamt nur wenige aus der Community
stammende dezidierte Forschungswerkzeuge gibt, die ohne tiefere
Vorkenntnisse beziehungsweise ohne Betreuung und Schulung nutzbar sind.
Entsprechend sind die Scholarly Makerspaces in jedem Fall als Ort der
Kompetenzvermittlung und des Angebots regelmäßiger Schulungen zu planen.
Im Sinne der Grundidee einer integrativen, interaktiven und
niedrigschwelligen Maker-Kultur ist außerdem eine Schwerpunktsetzung auf
Community-Bildung zu empfehlen, also das Angebot eines
Kommunikationsrahmens, in dem sich Nutzende mit unterschiedlichen
Interessen und Kompetenzstufen begegnen und austauschen können.
Scholarly Makerspaces werden in dieser Richtung als \emph{facilitator}
gedacht.

\hypertarget{interviews}{%
\subsection{2.2 Interviews}\label{interviews}}

Die im Projekt durchgeführten qualitativen Leitfadeninterviews mit einem
Convenience-Sample von 16 Expertinnen und Experten aus den Bereichen
Digital Humanities und Informationsinfrastrukturen dienten der
Anforderungsanalyse, um konkret feststellbare Bedarfe und wahrgenommene
Desiderate zu erheben. Zwar sind die Ergebnisse wegen des Zuschnitts der
Stichprobe nicht repräsentativ, jedoch erlauben sie aufgrund der
Fachkompetenz der Befragten tiefere Einblicke in die gegenwärtige
Ausgangslage sowie die Berücksichtigung kritischer Aspekte für die
Umsetzung von Scholarly Makerspaces.

Die befragten Personen zeigten sich durchgehend interessiert am Konzept
der Scholarly Makerspaces und begrüßten die aktive Rolle von
Universitätsbibliotheken bei der Vermittlung von
Digital-Humanities-Tools und entsprechenden Kompetenzen. Gleichzeitig
wurde mehrfach betont, dass eine deutliche Abgrenzung zu bestehenden
Digital-Humanities-Zentren beziehungsweise DH Labs erfolgen
sollte.\footnote{Vergleiche Cologne Center for eHumanities
  \url{http://www.cceh.uni-koeln.de/}; Göttingen Centre for Digital
  Humanities \url{http://www.gcdh.de/en/};
  Historisch-Kulturwissenschaftliche Informationsverarbeitung (Köln)
  \url{http://www.hki.uni-koeln.de/}; Kallimachos -- Zentrum für
  digitale Edition und quantitative Analyse (Würzburg)
  \url{http://kallimachos.de/kallimachos/index.php/Hauptseite};
  Computerlinguistik und Texttechnologie (Bielefeld)
  \url{http://www.uni-bielefeld.de/lili/studium/faecher/texttechnologie/arbeitsbereich.html};
  Trier Center for Digital Humanties
  \url{https://www.kompetenzzentrum.uni-trier.de/de/}; Austrian Centre
  for Digital Humanities (Graz)
  \url{https://informationsmodellierung.uni-graz.at/en/}.} Insbesondere
sollten Universitätsbibliotheken nicht mit der Entwicklung von digitalen
Forschungswerkzeugen befasst sein, da dies fachwissenschaftliche
Expertise und umfangreiche Ressourcen benötige. Vielmehr bestünde das
Potenzial von Scholarly Makerspaces in einer erhöhten Sichtbarkeit und
verbesserten Zugänglichmachung von DH-Tools und damit auch dem Erreichen
neuer Zielgruppen. Eine engere Kooperation und Arbeitsteilung zwischen
DH-Communities und Universitätsbibliotheken wurde in diesem Zusammenhang
als unabdingbar angesehen.

Einigkeit bestand auch über den Befund, dass die \emph{digital literacy}
allgemein sowohl bei den Studierenden als auch bei den Forschenden und
Dozierenden der Geistes- und Kulturwissenschaften vergleichsweise gering
ausgeprägt ist und stark von der persönlichen Motivation, dem Vorwissen
und der Lernbereitschaft abhängt. \emph{Digital literacy} bezieht sich
hierbei auf wissenschaftsspezifische digitale Anwendungen und
Strukturen, wozu unter anderem auch Informationsinfrastrukturen und
digitale Forschungsdaten zählen. Gerade in diesen Bereichen wird von
Universitätsbibliotheken eine aktive Vermittlungsrolle eingefordert. Die
Scholarly Makerspaces lassen sich daher als direkte Reaktion auf diese
Erwartung sehen.

Weiterhin findet eine systematische Methodenausbildung für
computergestützte und werkzeugbasierte digitale Verfahren im Vergleich
zu eher technikaffinen Wissenschaftsbereichen wie den MINT-Fächern kaum
statt. Scholarly Makerspaces können und sollen dieses Desiderat nicht
auffangen. Sie können jedoch einen Anlaufpunkt sowie Kommunikations- und
Koordinierungsraum für Akteure und Initiativen in diesem Bereich sein.
Darüber können Universitätsbibliotheken als Infrastruktur- und
Serviceanbieter einen Beitrag leisten, indem sie zielgruppenorientiert
und niedrigschwellig Angebote schaffen, die sich auch für den
Erstkontakt mit digitalen Forschungswerkzeugen eignen.

Folglich wird generell die Bedeutung der Scholarly Makerspaces als
Explorationsraum betont, in dem vorinstallierte Forschungswerkzeuge
sowie digitale Forschungsmaterialien in unterschiedlichen
Aufbereitungsstufen im Sinne einer Sandbox angeboten werden, um sich mit
den Möglichkeiten und Grenzen digitaler Verfahren praktisch und
ergebnisoffen auseinanderzusetzen. Für eine direkte Verbindung mit der
Lehre sind Implementierungsoptionen in vorhandene E-Learning-Umgebungen
(zum Beispiel Moodle an der Humboldt-Universität zu Berlin) vorgesehen.

Mehrfach wurde darauf hingewiesen, dass Scholarly Makerspaces einen
offen und flexibel nutzbaren physischen Raum anbieten sollten, in dem
physische Werkzeuge (wie Buchscanner, Objekt-Scanner, 3D-Drucker,
Plotter) zur Verfügung stehen, Schulungen und weitere Veranstaltungen
stattfinden können sowie ein direkter Kontakt vor Ort ermöglicht wird.

Bei der Auswahl der angebotenen digitalen Werkzeuge sollten transparente
Kriterien angewendet werden, um Nutzenden die Orientierung zu
erleichtern. Vorgeschlagen wurden unter anderem folgende Aspekte:

\begin{itemize}

\item
  Aktualität beziehungsweise Stand der Wissenschaft und Technik
\item
  Pflegegrad und Nachhaltigkeitskonzept
\item
  Akzeptanz in der Fach-Community
\item
  Installationsaufwand und Usability
\item
  Interoperabilität (zum Beispiel Exportformate, Datenstandards)
\item
  Ansprechpartner auf Seiten der Entwickler
\item
  Schulungsangebote durch die Scholarly Makerspaces.
\end{itemize}

Auch die konkrete lokale Nachfrage sollte berücksichtigt werden.
Allgemein wird ein Grundstock an vergleichsweise generischen und
exemplarischen Lösungen erwartet, der je nach lokaler Situation um
spezifische Lösungen ergänzt werden kann.

Allerdings wurde darauf hingewiesen, dass aufgrund der Vielzahl und
dynamischen Entwicklung von DH-Tools nicht das einzelne
Forschungswerkzeug im Fokus stehen sollte, sondern die entsprechende
Funktion von Werkzeugen innerhalb typischer Forschungsprozesse. Die
TaDi\-RAH-Taxonomie bietet hier eine erste Orientierung (beispielsweise
Erfassen, Erzeugen, Anreichern, Analyse, Interpretation, Speicherung,
Dissemination).\footnote{\url{https://github.com/dhtaxonomy/TaDiRAH/blob/master/deu/aktivitaeten.md}.}
Idealerweise bieten Scholarly Makerspaces zu jeder Funktion eine Auswahl
an Werkzeugen mitsamt Evaluation und Nutzungsbeschreibung, eventuell
sogar einem Tutorial an. Dabei sollten durchaus auch kommerzielle
Angebote berücksichtigt werden, die gegebenenfalls für die Hochschule
lizenziert werden können. Ein Beispiel wäre der in der DH-Community
weithin etablierte XML-Editor Oxygen.\footnote{\url{https://www.oxygenxml.com/}.}

Im Unterschied zu bloßen Linklisten für DH-Tools sollten Scholarly
Makerspaces die Angebote kuratieren und didaktisch aufbereiten. Ein
konkreter Vorschlag ist eine Art geführte Tour, die von Nutzenden auch
eigenständig absolviert werden kann. Diese Tour sollte einen
idealtypischen Arbeits- und Forschungsprozess mit den entsprechenden
werkzeugbasierten Verfahren abbilden, sodass sich Nutzende mit neuen
Forschungsverfahren über praktische Erfahrungen, beispielsweise über
Demos oder Testdaten, vertraut machen können. Am Beispiel der
Textanalyse würde eine solche geführte Tour beispielsweise folgende
Schritte umfassen:

\begin{itemize}

\item
  Erstellung eines Digitalisates,
\item
  Handschriftenerfassung beziehungsweise Texterfassung,
\item
  XML-Strukturierung,
\item
  Normalisierung,
\item
  Anreicherung beziehungsweise Annotation,
\item
  Textanalyse,
\item
  Visualisierung,
\item
  Export zur Ausgabevorbereitung in Druck und digital,
\item
  Export zur Langzeitarchivierung und Bereitstellung zur Nachnutzung.
\end{itemize}

Weiterhin wurde in den Interviews hervorgehoben, dass die Anbieter von
Scholarly Makerspaces über hinreichende DH-Kompetenzen verfügen müssen,
um Schulungen und Beratungen vor Ort durchführen beziehungsweise
koordinieren zu können. Da diese Kompetenzen oftmals nicht im
vorhandenen Personalbestand vorausgesetzt werden können, sollte bei
Personalentscheidungen auf entsprechende Qualifizierungen geachtet
werden, beispielsweise mit dem Profil eines \emph{Data Librarian}
beziehungsweise von Absolventinnen und Absolventen eines
DH-Studienganges.

Schließlich wurde empfohlen, die Umsetzung eines Scholarly Makerspaces
in das strategische Gesamtkonzept der jeweiligen Universitätsbibliothek
beziehungsweise Universität zu integrieren und bei der Einführung auf
Marketing-Strategien wie Branding und gezielte Öffentlichkeitsarbeit zur
Etablierung vor Ort zu setzen.

\hypertarget{expertenworkshop}{%
\subsection{2.3 Expertenworkshop}\label{expertenworkshop}}

In einer Diskussion mit Expertinnen und Experten im Rahmen des
FuReSH-Workshops \enquote{Scholarly Makerspaces -- Bibliotheken als
Vermittlungsplattform von Digital-Humanities-Tools}\footnote{Vergleiche
  Bericht zum Workshop im FuReSH-Projektblog:
  \url{https://blogs.hu-berlin.de/furesh/2019/03/12/bericht-zum-workshop-scholarly-makerspaces-bibliotheken-als-vermittlungsplattform-fuer-digital-humanities-tools/}.}
wurden weitere Aspekte konkretisiert und diskutiert. Beispielsweise
sollten Scholarly Makerspaces mit bereits bestehenden Serviceangeboten
der jeweiligen Hochschule integriert werden, sodass
Infrastrukturlösungen (zum Beispiel Cloud-Speicher, Umgebungen zum
kollaborativen Arbeiten, Medienrepositorien, Mediatheken,
Publikationsserver) und weitere digitale Werkzeuge (zum Beispiel
Literaturverwaltungsprogramme) bis hin zu physischen Werkzeugen (zum
Beispiel 3D-Drucker, Plotter) in einem größeren Zusammenhang sichtbar
werden.

Ein weiterer angesprochener Aspekt war die Orientierung an den Curricula
mit Schnittmengen zu digitaler Forschung an der jeweiligen Hochschule.
Da die Kompetenzvermittlung zur digitalen Forschung in den Kultur- und
Geisteswissenschaften vornehmlich die Aufgabe der Fach-Communities und
damit der einzelnen Fakultäten und Institute ist, sollten die
Infrastrukturangebote auf deren Aktivitäten und Bedarfe zugeschnitten
sein. An der Humboldt-Universität zu Berlin könnten beispielsweise für
Studierende der digitalen Geschichtswissenschaft entsprechende
Annotations- und Analysewerkzeuge auf arbeitsfähigen Workstations
bereitgestellt werden, die eine seminarbegleitende Methodenvermittlung
ermöglichen.

Des Weiteren wurde darauf hingewiesen, dass verschiedene Konzeptionen
von sogenannten Labs (Library Lab\footnote{Vergleiche British Library
  Labs \url{https://www.bl.uk/projects/british-library-labs}; ETH
  Library Lab \url{https://www.librarylab.ethz.ch/}; Harvard Library Lab
  \url{https://osc.hul.harvard.edu/liblab/}.}, DH Lab, Data Lab und
andere) zwar der Idee von Scholarly Makerspaces ähneln, jedoch andere
Schwerpunkte aufweisen und daher eher arbeitsteilig als Ergänzung oder
Partner verstanden werden sollten. Während es zum Beispiel bei dem
geplanten SBB Lab der Staatsbibliothek Berlin vorrangig um die
Kuratierung des eigenen digitalen Bestandes geht und bei der Konzeption
des Human-Centered Data Laboratory an der Freien Universität Berlin um
die kritische Reflexion über die Mensch-Maschine-Interaktion, steht bei
den Scholarly Makerspaces vor allem ein forschungsbegleitender und
didaktischer Ansatz im Vordergrund. Sowohl das SBB Lab als auch das Data
Laboratory können aber im Netzwerk der Scholarly Makerspaces aktive
Rollen übernehmen -- das erste, in dem es digitalisierte Quellen zur
Beforschung bereitstellt; das zweite, in dem es aktiv beispielsweise
über Veranstaltungen die Reflexion zu den Bedingungen digitaler
Forschung anregt.

In der Diskussion wurde auch erwogen, ob für die angebotenen Schulungen
in den Scholarly Makerspaces nicht auch als anrechenbare Leistung im
Rahmen des Studiums vergeben werden könnten, um einen höheren Anreiz für
Studierende zu schaffen. Allerdings gab es hierbei unterschiedliche
Meinungen. Da es nicht Aufgabe von Universitätsbibliotheken ist, über
Studienordnungen und Kreditierungen zu entscheiden, können solche
Erwägungen nur im Gespräch mit den entsprechenden Fakultäten und
Instituten substanziiert und allenfalls mittel- oder langfristig
umgesetzt werden. Der Schwerpunkt der Scholarly Makerspaces sollte nicht
die Übernahme einer Methodenausbildung sein, sondern ein Beratungs- und
Motivationsangebot für die reflexive und im Ergebnis
forschungsorientierte Beschäftigung mit digitalen Werkzeugen, Daten und
Verfahren.

Einigkeit bestand über die Herausforderung, dass sich die Umsetzung von
Scholarly Makerspaces an Universitätsbibliotheken in einem Spannungsfeld
zwischen personellen Ressourcen und der Komplexität der Aufgaben
befindet. Allgemein wurde die dauerhafte Betreuung eines Scholarly
Makerspaces mit ein bis zwei Vollzeitstellen als realistisch angesehen.

Für die Implementierungsphase sollten schon aus pragmatischen Gründen
Arbeitsschwerpunkte klar definiert und priorisiert werden. Für das
Umsetzungsmodell wurde daher eine Differenzierung zwischen festen
Aufgaben im Sinne der Grundidee und variablen Schwerpunkten in
Abstimmung mit den jeweiligen lokalen Bedingungen und Anforderungen
diskutiert. Zu den festen Diensten zählt neben der Kuratierung digitaler
Angebote und der Bereitstellung eines Raumes mit anwendungsbereiten
Workstations, der zugleich Schulungs-, Explorations-, Kommunikations-
und Reflexionsraum sein soll, vor allem der Schwerpunkt Vernetzung und
Kooperation mit weiteren Partnern. Somit muss im Idealfall gerade bei
Beratungen keine vollumfassende Expertise vor Ort vorhanden sein.
Wichtig ist, den Beratungsbedarf eindeutig zu bestimmen und die
passenden Expertinnen und Experten vermitteln zu können.

\hypertarget{konkretisierung-des-zielszenarios}{%
\section{3 Konkretisierung des
Zielszenarios}\label{konkretisierung-des-zielszenarios}}

\hypertarget{eingrenzung-und-beschreibung-des-angebots}{%
\subsection{3.1 Eingrenzung und Beschreibung des
Angebots}\label{eingrenzung-und-beschreibung-des-angebots}}

Als zentrale Herausforderung wurde in der Erhebung festgestellt, dass
ein Angebot wie die Scholarly Makerspaces zu stark in Richtung der
Fachwissenschaften und insbesondere der Digital Humanities streben
könnte und damit das eigentliche Angebots- und Leistungsprofil von
Universitätsbibliotheken überschreiten würde. Daher müssen der
bibliothekarische Hintergrund und der kollaborative Ansatz des Angebotes
klar herausgestellt werden. Hinter der Idee der Scholarly Makerspaces
steht die Prämisse, dass eine Universitätsbibliothek ihre Nutzenden
grundsätzlich befähigen muss, die für die Wissenschaft notwendigen
Bestände und bibliothekarisch vermittelten Bezugsmaterialien angemessen
rezipieren zu können.

In dem Maße, in dem die lange dominierenden narrativen
Publikationsformen (zum Beispiel Zeitschriftenaufsatz, Monografien,
Sammelbände) erweitert beziehungsweise digital überformt werden (zum
Beispiel Datenbanken, Forschungsdatenpublikationen, Enhanced
Publications, Korpora) und oft nur mit bestimmten Anzeige-, Zugangs- und
Verarbeitungsmitteln genutzt werden können, muss die Bibliothek folglich
Angebote schaffen, die den Nutzenden eine grundlegende Anwendung dieser
Mittel ermöglicht.

Hinzu kommt, dass die Auseinandersetzung mit derartigen Materialien
nicht mehr in einer linear modellierbaren reinen Rezeption erfolgt,
sondern häufig über dynamische Interaktion. Ein Desiderat bleibt dabei,
dass die Nutzung solcher Formen der digitalen Forschung bestimmte neue
Literarizitäten (zum Beispiel \emph{digital literacy}, \emph{data
literacy, tool literacy}) voraussetzt. Diese beschränken sich nicht auf
die reine Bedienungsfähigkeit, sondern umfassen den eigentlich
wichtigeren Aspekt eines Verstehens der Funktionsweise, Möglichkeiten
und Grenzen digitaler Forschungsverfahren und -technologien. Dies wird
als Forschungsagenda teilweise in der Bibliotheks- und
Informationswissenschaft und in den Konzepten der Science Technology
Studies abgedeckt, ist jedoch von derart grundlegender Relevanz, dass
entsprechende Reflexionsschritte Teil jeder digital beeinflussten
Methodologie sein sollten. Dieser Schritt soll und muss in der Tiefe den
Fachwissenschaften überlassen werden. Da sich diese jedoch, wie auch die
Erhebung ergab, bislang sehr heterogen, oft nicht systematisch, teils
auch sogar nur implizit mit diesen Fragen auseinandersetzen, ergibt sich
für Universitätsbibliotheken als Dienstleister der wissenschaftlichen
Informationsversorgung die zumindest vorübergehende Aufgabe, gezielt
angleichende Vermittlungen von Basiskompetenzen zur Nutzung digitaler
Formen wissenschaftlicher Information und Kommunikation anzubieten
beziehungsweise anzuregen.

Für die Forschenden der Hochschule umfasst das Angebot neben einer
ebenfalls nachgefragten allgemeinen Kompetenzvermittlung auch eine
Beratung mit dem Ziel der Forschungsoptimierung und -qualitätssicherung.
In Entsprechung zu bereits vorhandenen Beratungsdiensten zum
Forschungsdatenmanagement und zum elektronischen Publizieren schließen
Scholarly Makerspaces die Lücke für die Auswahl von Verfahren und
Werkzeugen. Sie helfen sicherzustellen, dass Forschende für ihre
Projekte den jeweils aktuellen Stand entsprechender Lösungen und
Standards kennen und berücksichtigen. Die Scholarly Makerspaces sind
demnach nicht nur Arbeits- und Trainingsort, sondern auch Anlaufpunkt
für einen systematischen Überblick zum aktuellen Entwicklungsstand zu
digitalen Forschungsverfahren und -werkzeugen für die Kultur- und
Geisteswissenschaften, etwa in Entsprechung zur Rolle von
Fachreferentinnen und -referenten.

\hypertarget{serviceangebote}{%
\subsection{3.2 Serviceangebote}\label{serviceangebote}}

Das Serviceportfolio von Scholarly Makerspaces bietet infrastrukturelle
Lösungen sowie Angebote zur Schulung, Beratung und Community-Bildung. Zu
den Infrastrukturangeboten gehören ein physischer Raum sowie die
Workstations mit einsatzbereiten digitalen Werkzeugen beziehungsweise
Schnittstellen zu weiteren digitalen Ressourcen. Die Angebote der
Schulung und Beratung umfassen neben einer Webpräsenz (zum Beispiel
Webseite, Weblog, Wiki) eingebundene oder verlinkte Tutorials in
verschiedenen medialen Formen sowie die direkte Kommunikation in Form
von Kompetenzschulungen, Beratungen und weiteren Veranstaltungen vor
Ort. Die Community-Bildung wird aktiv durch die Etablierung und Pflege
eines Netzwerkes unterstützt. Die Angebote sind so konzipiert, dass sie
sich ergänzen und in Wechselwirkung stehen.

\hypertarget{zielgruppen}{%
\subsection{3.3 Zielgruppen}\label{zielgruppen}}

Allgemein sollen sich in den unmittelbaren Zielgruppen potenziell alle
Nutzenden der Universitätsbibliothek finden können. Für die
Operationalisierung und den Zuschnitt entsprechender Angebote und
Kommunikationsstrategien sind diese jedoch konkreter zu differenzieren.
Neben der Gruppe der allgemein Interessierten sind dies im Bereich der
Geistes- und Kulturwissenschaften vor allem:

\pagebreak
\begin{itemize}
\item Forschende,
\item Projektplanende,
\item Lehrende,
\item Studierende.
\end{itemize}

Dabei wird ein Schwerpunkt besonders bei der Kooperation mit der
jeweiligen Einrichtung gezielt auf der Nachwuchsförderung liegen, da
hier aus den Erhebungen das größte Vernetzungspotenzial und zugleich
eine besonders ausgeprägte thematische Aufgeschlossenheit und Dynamik
ermittelt werden konnte.

Mittelbare Zielgruppen sind zudem Stakeholder, für die Scholarly
Makerspaces selbst ein Mittel zum Entwicklungsmonitoring, zur
Bedarfsermittlung und zur Kommunikation werden:

\begin{itemize}

\item
  Anbieter von Werkzeugen und Verfahren (zum Beispiel Showcasing,
  Schulungen, Testläufe),
\item
  Akteure der Forschungsplanung\footnote{Zum Beispiel
    Forschungsabteilungen, Sonderforschungsbereiche und
    Exzellenzcluster.} (zum Beispiel Antragstellung für Drittmittel),
\item
  Akteure aus der Hochschuladministration zur Bedarfsfeststellung und
  Hochschulentwicklung.
\end{itemize}

Langfristig ist auch eine Erweiterung des Angebots auf außerakademische
Zielgruppen angedacht:

\begin{itemize}

\item
  Wissenschaftsvermittlung,
\item
  Bürgerwissenschaft,
\item
  Kreativ- und Digitalwirtschaft.
\end{itemize}

Weitere Zielgruppen bilden diverse und in ihren Interessen heterogene
externe Akteure aus dem breiten Feld der digitalen Auseinandersetzung
mit Kulturobjekten.\footnote{Ein Kontaktpunkt wäre dafür zum Beispiel
  das Coding-Da-Vinci-Projekt beziehungsweise die daran beteiligten
  Akteure siehe \url{https://codingdavinci.de/}.} Dem offenen und
inklusiven Prinzip der Makerspaces folgend wäre auch denkbar, dies im
Rahmen von Vermittlungsaktivitäten eines \emph{public understanding of
science} besonders zu fördern. Dieser Aspekt wurde in den Interviews
mehrfach erwähnt. Allerdings berührt er die prinzipielle Ausrichtung der
Universitätsbibliothek beziehungsweise der Hochschule im Sinne eines
Leitbilds. Er soll daher hier nur als Option benannt, jedoch vorerst
nicht weiter differenziert werden. Gleiches gilt für die in Interviews
ausdrücklich angeregte Kooperation mit der lokalen Kreativ- und
Digitalwirtschaft. Auch hierfür müssten mögliche Kontakt- und
Kooperationsszenarien gezielt erstellt werden, deren Detaillierung
jedoch nicht im unmittelbaren Fokus der Konzeptstudie liegt.

\hypertarget{nutzungsszenarien}{%
\subsection{3.4 Nutzungsszenarien}\label{nutzungsszenarien}}

Unabhängig vom jeweils konkreten Anwendungsfall wird zur besseren
Strukturierung der Materialien als Grundlage der Kompetenzbewertung ein
vierstufiges Vermittlungskonzept gewählt. Es bezieht sich jeweils auf
Verfahren und Werkzeuge. Die vier Stufen sind:

\begin{enumerate}
\def\labelenumi{\arabic{enumi}.}

\item
  \emph{Kennenlernen} -- mit einem allgemeinen, unspezifischen Interesse
  als Motivation,
\item
  \emph{Lernen} -- mit der Motivation, bestimmte Verfahren
  beziehungsweise Werkzeuge gezielt zu lernen,
\item
  \emph{Forschungsvorbereitung} -- Auswählen, Beurteilen, Anpassen von
  Verfahren bzw. Werkzeugen im Zuge der Planung eines
  Forschungsprojektes,
\item
  \emph{Forschung} -- Durchführung von direkten Forschungsschritten in
  Scholarly Makerspaces oder in Begleitung durch die Angebote der
  Scholarly Makerspaces.
\end{enumerate}

Für jede dieser Stufen gibt es spezifische Anforderungen an die
Vermittlungs- und Kommunikationsstrategien, also die Aufbereitung der
Materialien. Die Differenzierung ermöglicht es, Anfragen, Annäherungen,
Ziele und Bedarfe vorzuordnen, um sie im Anschluss angemessen
adressieren zu können. Damit wird das Ziel einer weitreichenden
Inklusivität erreicht. Die Angebote dieser Stufen sollen den Nutzenden
auch helfen, eigene Ziele, Bedarfe und Kompetenzen klarer bestimmen zu
können.

\hypertarget{digitale-geistes--und-kulturwissenschaftliche-forschung}{%
\subsubsection{Digitale geistes- und kulturwissenschaftliche
Forschung}\label{digitale-geistes--und-kulturwissenschaftliche-forschung}}

Aus den Interviews und der Analyse der jeweiligen Forschungsfelder ergab
sich, dass eine Orientierung auf die Texterschließung und
-strukturierung einer der primären Ansatzpunkte für das Aufsetzen eines
solchen Angebotes darstellt. Dies erklärt sich unter anderem dadurch,
dass in diesen Bereichen ein besonders hoher Entwicklungsgrad sowie eine
breite und sehr aktive Community vorliegen. So ist TEI-XML ein weithin
anerkannter Standard zur Textauszeichnung. Dies wirkt sich folgerichtig
auf Workflows und Analyseverfahren sowie auf die Werkzeugentwicklung
aus. Beispielsweise liegt mit dem Oxygen XML-Editor eine stabile Lösung
vor, die auch in den Interviews fast durchgängig als essenziell benannt
wurde. Zugleich sind die Komplexität und der Anspruch an die
Datenhaltung und -verarbeitung bei Texten im Vergleich zu anderen
medialen Repräsentationsformen weniger ausgeprägt.

Als Bezug für die Angebote der Scholarly Makerspaces wird daher ein
idealtypischer Verarbeitungsvorgang von Textmaterialien beginnend bei
der Digitalisierung über die Textauszeichnung und Erschließung bis zur
Ausgabe zum Beispiel als Edition gewählt.\footnote{Vergleiche das
  Beispiel einer Prozesskette für die digitale Textanalyse im Abschnitt
  „Interviews".} Die Prozesskette ist nicht geschlossen, sondern
berücksichtigt diverse Ein- und Ausstiegspunkte. Sie dient zugleich
dazu, einen weitgehend lückenlosen Ablauf zu erfassen. Die Schwerpunkte
liegen dabei auf den bibliothekarischen und
bibliothekswissenschaftlichen Kompetenzfeldern des Information
Retrievals, der Datenaufbereitung, -erschließung, -dissemination und
-archivierung. Für die spezifisch fachlichen Aspekte werden in
Kooperation mit Fach-Communities grundlegende Einführungen entwickelt.
Parallel ist angedacht, im Rahmen der Exportschnittstellen auch die Idee
der offenen Wissenschaft und damit der offenen Bereitstellung von Daten
zu unterstützen. Die tiefergehende Beratung muss über Expertinnen und
Experten mit jeweils dezidiert fachlichem Hintergrund erfolgen. Hierfür
wird ein Kontaktnetzwerk aufgebaut und gepflegt. Die Aufgabe der
Scholarly Makerspaces als Bibliotheksangebot liegt an dieser Stelle in
der Vermittlung von Expertise entweder über Kontakte oder auch bei
breiterem Bedarf über die Organisation und Durchführung von
Vermittlungsveranstaltungen mit Expertinnen und Experten.

Nach dem Vorbild textorientierter digitaler Forschung werden auch für
bild-, objekt-, und raumorientierte sowie andere multimodale und
multimediale Forschungsformen ähnliche Ansätze aufgebaut. Sofern
Lösungen ermittelbar sind, wird auf diese verwiesen und gegebenenfalls
ein Kontakt vermittelt. Inwieweit auf Nachfrage auch stärker
spezialisierte Werkzeuge auf den Workstations installiert und
gegebenenfalls auch lizenziert werden können beziehungsweise sollen, ist
von dem jeweiligen Bedarf der Forschungseinrichtung abhängig.

\hypertarget{beratung-und-unterstuxfctzung-fuxfcr-die-lehre}{%
\subsubsection{Beratung und Unterstützung für die
Lehre}\label{beratung-und-unterstuxfctzung-fuxfcr-die-lehre}}

Für das Ziel einer nachhaltigen Vermittlung vor allem hinsichtlich der
Zielgruppe der Nachwuchsforschenden, insbesondere Promovierenden,
leisten Scholarly Makerspaces eine aktiv unterstützende Rolle bei
entsprechenden Lehrangeboten. Dies wurde ausdrücklich in den Interviews
als Desiderat benannt. Ein solches Angebot kann jedoch nur kooperativ
mit den jeweiligen die Lehre anbietenden Instituten sowie Dozierenden
realisiert werden. Das Vermittlungs- und Beratungsangebot umfasst für
diesen Zweck zwei Portfolios -- ein infrastrukturelles und ein
didaktisches. Im infrastrukturellen Portfolio werden bestehende
Infrastrukturangebote\footnote{Für die Humboldt-Universität zu Berlin
  beispielsweise die HU-Box, das Medienrepositorium, die Angebote des
  edoc-Servers.} sowie die Angebote der Scholarly Makerspaces (zum
Beispiel Raum, Workstations, digitale und physische Werkzeuge,
Retrieval-Möglichkeiten) abgebildet. Ziel ist, die Dienste im Rahmen des
Lehrangebots als Bausteine digitaler Forschung zu vermitteln und damit
eventuelle Nutzungshürden gezielt abzubauen. Für die didaktischen
Angebote wird eine Best-Practice-Sammlung für in der Lehre besonders
geeignete digitale Werkzeuge und Verfahren zusammengestellt und
Lehrenden zur Nachnutzung bereitgestellt. Dabei wird mit
institutionellen Akteuren wie CLARIAH kooperiert. Lehrende können also
über die Scholarly Makerspaces grundlegende Materialien für die Planung
und Durchführung von Lehrveranstaltungen abrufen sowie den Raum der
Scholarly Makerspaces direkt nutzen oder den Studierenden zur
veranstaltungsbegleitenden individuellen oder Gruppenarbeit anbieten.
Idealerweise werden diese Angebote in bestehende E-Learning-Plattformen
eingebunden.

\hypertarget{scholarly-makerspaces-als-raum}{%
\subsubsection{Scholarly Makerspaces als
Raum}\label{scholarly-makerspaces-als-raum}}

Eine Stärke der Scholarly Makerspaces liegt darin, dass sie zieloffen
nutzbar sind. In diesem Sinne wird das Prinzip der Serendipität
unterstützt, bei dem sich aus einer offenen Auseinandersetzung mit einem
Inhalt oder hier auch den Möglichkeiten eines Werkzeugs oft ungeplant
neue Ansätze, Einsichten, Impulse und Lösungen ergeben. Dies soll zudem
über den Community-Aspekt, also den Makerspace als Begegnungs- und
Kommunikationsraum bewusst gefördert werden. Die Aufgabe der Betreiber
ist dabei, entsprechende Prozesse zu unterstützen, zu moderieren und
gegebenenfalls zu verstärken und zu dokumentieren. Als Einstieg liegt
der Schwerpunkt auf der oben benannten Stufe des „Kennenlernens".

Es soll jedoch ausdrücklich auch die Möglichkeit konkreter Forschungs-
und Spin-Off-Ansätze berücksichtigt und gefördert werden. Neben der
Vermittlung und Beratung sind Anregung und Unterstützung, Steuerung von
Reflexionsprozessen über Aspekte digital geprägter Kultur- und
Geisteswissenschaften zentrale Anliegen der Scholarly Makerspaces. Sie
umfassen eine technologische Perspektive ebenso wie
wissenschaftssoziologische, -ökonomische und -ethische Aspekte,
medientheoretische Überlegungen und epistemologische Fragen. In den
Erhebungen zur Studie wurde die Erwartung formuliert, dass gerade über
das Zusammenwirken von Akteuren aus Universitätsbibliotheken, der
Informationsinfrastruktur, den Digital Humanities und kultur- und
geisteswissenschaftlichen Communities neue methodologische und
technologische Perspektiven eröffnet werden können.

Einen besonderen Schwerpunkt übernimmt die Förderung eines
interdisziplinären Dialogs. Perspektivisch soll dieser auch
außerakademische Zielgruppen einschließen und somit eine stärkere
Verbindung zwischen Wissenschaft und Gesellschaft fördern. Die Scholarly
Makerspaces stellen für diese Kommunikationen informelle und stärker
formalisierte Strukturen bereit. Der informelle Austausch kann über
entsprechende Rahmungen wie beispielsweise die Organisation von Barcamps
oder Hackathons unterstützt werden.

Zugleich werden Fragestellungen und Entwicklungen mit Zuschnitt auf
besondere Forschungs- und Erkenntnisinteressen über regelmäßige
Veranstaltungen adressiert. Ausgewählte Ergebnisse der Reflexionsarbeit
sollen in angemessener Form dokumentiert und publiziert werden. Die
Scholarly Makerspaces betreiben zu diesem Zweck ein Weblog.\footnote{Weblog
  des FuReSH-Projekts:
  \url{https://blogs.hu-berlin.de/furesh/2018/05/14/scholarly-makerspaces/}.}

\hypertarget{scholarly-makerspaces-als-kooperationspartner-der-dh-forschung}{%
\subsubsection{Scholarly Makerspaces als Kooperationspartner der
DH-Forschung}\label{scholarly-makerspaces-als-kooperationspartner-der-dh-forschung}}

Von Seiten der befragten Akteure aus dem Bereich der Tool-Entwicklung in
den Digital Humanities wurde ein großes Potenzial für das Angebot im
direkten Kontakt zu lokalen Communities mit jeweils spezifischen
Ansprüchen gesehen. Die Scholarly Makerspaces bieten diesen Anbieter-
und Entwicklergruppen den Raum, um neue Werkzeuge und Lösungen gezielt
zu vermitteln oder auch testweise anzubieten und zu evaluieren.

Scholarly Makerspaces werden also gerade nicht als \emph{Labs} im Sinne
von DH-Labs, Digital-Huma\-nities-Zentren oder der an vielen Stellen
entstehenden Library Labs verstanden. Ihr inhaltlicher Schwerpunkt liegt
vor allem in der grundlegenden Kompetenz- und Zugangsvermittlung im
Sinne der oben ausgeführten Beschreibung. Es geht also primär um
Vermittlungsstrategien und Möglichkeiten einer stärkeren und
bedarfsnäheren Zusammenführung von Tool-Entwicklung und Tool-Nutzung. Im
Organisationsmodell sind dafür Grundlagen in den Bereichen
\emph{Monitoring} und \emph{Netzwerk} zu schaffen.

\hypertarget{antrags--und-forschungsberatung}{%
\subsubsection{Antrags- und
Forschungsberatung}\label{antrags--und-forschungsberatung}}

Ein weiterer Baustein und eine Erweiterung der Ursprungsidee liegt in
der Operationalisierung des in den Scholarly Makerspaces vorhandenen
Wissens zu Entwicklungen und Möglichkeiten digitaler Forschung mit dem
Ziel einer Optimierung der Forschungsplanung. Hierbei wird direkt das
Desiderat adressiert, dass Entscheidungen zur Tool-Auswahl bei
Forschungsprojekten und -planungen häufig nicht auf Basis einer
systematischen Übersicht, sondern in Rückgriff auf subjektives
Erfahrungswissen getroffen werden. Scholarly Makerspaces helfen, dieses
zu kontextualisieren und auf diesem Weg auch Vernetzungs-,
gegebenenfalls sogar Kooperationsmöglichkeiten und denkbare Alternativen
aufzuzeigen sowie Machbarkeits- und Komplexitätsabschätzungen zu
unterstützen.

\hypertarget{ausblick}{%
\section{4 Ausblick}\label{ausblick}}

Die Laufzeit dieser Konzeptstudie ist bis Ende Oktober 2019 vorgesehen.
Idealerweise schließt sich ein Projekt zur prototypischen
Implementierung eines Scholarly Makerspaces an der
Universitätsbibliothek der Humboldt-Universität zu Berlin zeitnah an.

In den nächsten Schritten werden Berichte zur Kostenstruktur sowie zum
Perspektivkonzept erarbeitet. Für die Kostenstruktur wird anhand eines
idealtypischen Szenarios ein konkretes Kostenmodell erstellt, das als
Orientierungsgrundlage sowohl für die Implementierung an der
Humboldt-Universität zu Berlin als auch für andere
Universitätsbibliotheken dienen soll. Im Perspektivkonzept werden zudem
die Nachnutzungsoptionen für andere Einrichtungen sowie Aspekte der
Nachhaltigkeit über drittmittelgeförderte Implementierungsphasen hinaus
erörtert.

Ein weiterer Bericht betrifft die Konkretisierung der
Kooperationsstrukturen. Ausgangspunkt sind hierbei die
FuReSH-Projektpartner einschließlich Vertreterinnen und Vertretern der
Forschungsverbünde DARIAH und CLARIN. Zusätzlich werden weitere
Kooperationsmöglichkeiten vorrangig mit lokalen Institutionen und
Netzwerken der Berliner DH-Community geprüft und integriert. In diesem
Zusammenhang wird ein weiterer Workshop stattfinden, bei dem das
bisherige Konzept der Scholarly Makerspaces vorgestellt und um weitere
Kooperationen geworben wird.

Schließlich wird das Konzept des Integrierten Monitoring erstellt, mit
dem Erkenntnisse des Projektes systematisch fortlaufend dokumentiert,
aufbereitet und kommuniziert werden. Da auch für die
Implementierungsphase ein integriertes Monitoring im Sinne eines
Beobachtungs- und Wissensmanagements stattfinden soll, werden zudem
entsprechende Definitionen von Beob\-achtungs- und Analyseprozessen
erarbeitet.

Bislang befindet sich das FuReSH-Projekt im Zeitplan. Nach einer ersten
Bilanz wird die Umsetzbarkeit von Scholarly Makerspaces an der
Humboldt-Universität zu Berlin positiv bewertet. Im Zuge einer
deutlicheren Abgrenzung von Digital-Humanities-Zentren und Konzepten
sogenannter Library Labs gewinnt die Idee der Scholarly Makerspaces
deutlich an Kontur. Insbesondere die spezifischen Zielgruppen und
Nutzungsszenarien sind ein Alleinstellungsmerkmal für
Universitätsbibliotheken, das folglich passgenaue Lösungen erfordert, um
digitale Forschung in den Geistes- und Kulturwissenschaften effektiv und
zukunftsorientiert zu unterstützen.

%autor

\begin{center}\rule{0.5\linewidth}{\linethickness}\end{center}

\textbf{Ben Kaden} ist Bibliothekswissenschaftler und Mitherausgeber von
LIBREAS. Er beschäftigt sich aktuell an der Universitätsbibliothek der
Humboldt-Universität zu Berlin mit der Entwicklung von Möglichkeiten,
einer Optimierung von Bibliotheksangeboten für die digital geprägte
Forschung. ORCID iD: \url{https://orcid.org/0000-0002-8021-1785}

\textbf{Michael Kleineberg} arbeitet im FuReSH-Projekt an der
Universitätsbibliothek der Humboldt-Universität zu Berlin. ORCID iD:
\url{https://orcid.org/0000-0002-6313-6795}

\end{document}
