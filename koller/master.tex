\documentclass[a4paper,
fontsize=11pt,
%headings=small,
oneside,
numbers=noperiodatend,
parskip=half-,
bibliography=totoc,
final
]{scrartcl}

\usepackage{synttree}
\usepackage{graphicx}
\setkeys{Gin}{width=.4\textwidth} %default pics size

\graphicspath{{./plots/}}
\usepackage[ngerman]{babel}
\usepackage[T1]{fontenc}
%\usepackage{amsmath}
\usepackage[utf8x]{inputenc}
\usepackage [hyphens]{url}
\usepackage{booktabs} 
\usepackage[left=2.4cm,right=2.4cm,top=2.3cm,bottom=2cm,includeheadfoot]{geometry}
\usepackage{eurosym}
\usepackage{multirow}
\usepackage[ngerman]{varioref}
\setcapindent{1em}
\renewcommand{\labelitemi}{--}
\usepackage{paralist}
\usepackage{pdfpages}
\usepackage{lscape}
\usepackage{float}
\usepackage{acronym}
\usepackage{eurosym}
\usepackage[babel]{csquotes}
\usepackage{longtable,lscape}
\usepackage{mathpazo}
\usepackage[normalem]{ulem} %emphasize weiterhin kursiv
\usepackage[flushmargin,ragged]{footmisc} % left align footnote
\usepackage{ccicons} 
\setcapindent{0pt} % no indentation in captions

%%%% fancy LIBREAS URL color 
\usepackage{xcolor}
\definecolor{libreas}{RGB}{112,0,0}

\usepackage{listings}

\urlstyle{same}  % don't use monospace font for urls

\usepackage[fleqn]{amsmath}

%adjust fontsize for part

\usepackage{sectsty}
\partfont{\large}

%Das BibTeX-Zeichen mit \BibTeX setzen:
\def\symbol#1{\char #1\relax}
\def\bsl{{\tt\symbol{'134}}}
\def\BibTeX{{\rm B\kern-.05em{\sc i\kern-.025em b}\kern-.08em
    T\kern-.1667em\lower.7ex\hbox{E}\kern-.125emX}}

\usepackage{fancyhdr}
\fancyhf{}
\pagestyle{fancyplain}
\fancyhead[R]{\thepage}

% make sure bookmarks are created eventough sections are not numbered!
% uncommend if sections are numbered (bookmarks created by default)
\makeatletter
\renewcommand\@seccntformat[1]{}
\makeatother


\usepackage{hyperxmp}
\usepackage[colorlinks, linkcolor=black,citecolor=black, urlcolor=libreas,
breaklinks= true,bookmarks=true,bookmarksopen=true]{hyperref}
\usepackage{breakurl}

%meta
%meta

\fancyhead[L]{C. Koller\\ %author
LIBREAS. Library Ideas, 35 (2019). % journal, issue, volume.
\href{http://nbn-resolving.de/}
{}} % urn 
% recommended use
%\href{http://nbn-resolving.de/}{\color{black}{urn:nbn:de...}}
\fancyhead[R]{\thepage} %page number
\fancyfoot[L] {\ccLogo \ccAttribution\ \href{https://creativecommons.org/licenses/by/4.0/}{\color{black}Creative Commons BY 4.0}}  %licence
\fancyfoot[R] {ISSN: 1860-7950}

\title{\LARGE{Weder Zensur noch Propaganda: Der Umgang des Schweizerischen Sozialarchivs mit rechtsextremem Material}}% title
\author{Christian Koller} % author

\setcounter{page}{1}

\hypersetup{%
      pdftitle={Weder Zensur noch Propaganda: Der Umgang des Schweizerischen Sozialarchivs mit rechtsextremem Material},
      pdfauthor={Christian Koller},
      pdfcopyright={CC BY 4.0 International},
      pdfsubject={LIBREAS. Library Ideas, 35 (2019).},
      pdfkeywords={Rechte Literatur, Bibliothekspraxis, Zürich, Schweizerisches Sozialarchiv, Bestandsprofil},
      pdflicenseurl={https://creativecommons.org/licenses/by/4.0/},
      pdfcontacturl={http://libreas.eu},
      baseurl={http://libreas.eu},
      pdflang={de},
      pdfmetalang={de}
     }



\date{}
\begin{document}

\maketitle
\thispagestyle{fancyplain} 

%abstracts
\begin{abstract}
\textbf{Kurzfassung}: Das Schweizerische Sozialarchiv hat eine lange
Tradition der Sammlung rechtsextremen Materials. Der Artikel skizziert
Umfang und Schwerpunkte dieser Sammlung und diskutiert
Erwerbungsstrategien und Benutzungsbedingungen im Lichte der Frage, ob
Bibliotheken solche Literatur \enquote{neutral} behandeln oder wegen politischer
Kontaminationsgefahr aus ihren Beständen fernhalten sollen.

\begin{center}\rule{0.5\linewidth}{\linethickness}\end{center}

\textbf{Abstract}: The Swiss Social Archives have a long tradition of
collecting far-right material. This article considers extent and
contents of this collection and discusses acquisition strategies and
terms of use in view of the question whether libraries ought to treat
such material \enquote{neutrally} or should rather exclude them to prevent
spreading anti-democratic ideas.
\end{abstract}

%body
\hypertarget{einleitung}{%
\section*{Einleitung}\label{einleitung}}

Im Sommer 2017 war das Buch \emph{Finis Germania} bei amazon.de
wochenlang auf Platz 1 der Bestsellerliste. Das im neurechten
Antaios-Verlag publizierte Werk des zuletzt an der Universität
St.~Gallen tätigen deutschen Historikers Rolf Peter Sieferle gab bei der
Veröffentlichung nach dessen Suizid sowohl wegen des von Kritikern als
rechtsextrem und antidemokratisch eingestuften Inhalts zu reden als auch
wegen Kontroversen um seine Präsenz in verschiedenen Sachbuchlisten.
Weniger Diskussionen verursachte die (Nicht-)Erwerbung des Bestsellers
durch wissenschaftliche Bibliotheken. Vom schweizerischen Nebis-Verbund,
dem mit rund 130 angeschlossenen Institutionen (darunter mehrere der
größten wissenschaftlichen Bibliotheken) bedeutendsten
Bibliotheksnetzwerk des Landes, fand das Buch einzig Eingang in den
Bestand des Schweizerischen Sozialarchivs in Zürich. Dies, obwohl
Sieferles frühere Werke von einer ganzen Reihe von Bibliotheken erworben
worden waren.

Diese Beobachtung führt zur Frage, inwiefern das Schweizerische
Sozialarchiv mit extremistischer Literatur einen speziellen Umgang
pflegt. Die seit 1906 bestehende Institution ist eine bundesanerkannte
Forschungsinfrastruktureinrichtung und umfasst ein Spezialarchiv, eine
wissenschaftliche Spezialbibliothek, eine aktualitätsbezogene
Sachdokumentation und einen Forschungsfonds. Seit ihren Anfängen wird
sie von einem überparteilichen Verein getragen und von der öffentlichen
Hand subventioniert. Die maßgebliche Gründerfigur, der
sozialreformerische Pfarrer und Politiker Paul Pflüger, ließ sich vom
\enquote{Musée social} inspirieren, einem 1894 in Paris entstandenen
sozialwissenschaftlichen Thinktank mit Museum, Bibliothek und
Forschungszentrum. Vor dem Hintergrund zunehmender sozialer Konflikte um
die Wende zum 20.~Jahrhundert sollte eine ähnliche Institution auch in
der Schweiz Wissen im Bereich der \enquote{sozialen Frage} im Dienste
reformerischen Handelns bereitstellen und damit zum gesellschaftlichen
Ausgleich beitragen. Gemäß aktuellem Leitbild dokumentiert das
Sozialarchiv \enquote{den gesellschaftlichen, politischen und
kulturellen Wandel vom 19. Jahrhundert bis in die Gegenwart mit Fokus
auf der Schweiz. Im Zentrum steht die Dokumentation der sozialen
Bewegungen als Motoren und Produkte dieser Veränderungsprozesse}
(Schweizerisches Sozialarchiv 2018). Dies geschieht durch koordinierte
Sammlung analoger und digitaler, schriftlicher und audiovisueller
Materialien in den drei Abteilungen Archiv, Bibliothek und
Sachdokumentation. Dank des Umstandes, dass die Sammlungstätigkeit des
Sozialarchivs nie durch Kriege oder Diktaturen beeinträchtigt worden
ist, verfügt es über einen auch im internationalen Vergleich
einzigartigen Bestand.

\hypertarget{lange-sammeltradition-rechtsextremen-materials}{%
\section*{Lange Sammeltradition rechtsextremen
Materials}\label{lange-sammeltradition-rechtsextremen-materials}}

Seit Beginn zielte die Sammlungsstrategie des Sozialarchivs auf
Berücksichtigung des gesamten politischen Spektrums, inklusive der
antidemokratischen Ränder, ab. Auch wenn in der Schweiz die
Nichtanschaffung eines Buches nicht gegen das Recht auf
Informationsbeschaffung oder andere Aspekte der
Meinungsäußerungsfreiheit der Benutzenden verstößt (Künzle 1992: 236),
bemühte sich das Sozialarchiv stets um Dokumentation aller politischen
Strömungen. So konterte Vorsteher Sigfried Bloch bereits 1921 die Kritik
eines Vorstandsmitglieds, es werde zu viel kommunistische Literatur
erworben, damit, dass ein wissenschaftliches Institut Material aus allen
Richtungen sammeln müsse (Häusler 2006: 9). In den 30er und 40er Jahren
gehörte die Institution zu den ganz wenigen im deutschsprachigen Raum,
die nebeneinander Bücher der NS-Propaganda und antifaschistische
Literatur in ihren Regalen stehen hatten. Nicht zuletzt dies zog bis
1945 eine rege Benutzung durch ExilantInnen aus Deutschland, Österreich
und Italien nach sich (Koller 2015: 386--390).

Von Hitlers \enquote{Mein Kampf} wurde erstmals eine Auflage von 1930
erworben. Eingang in den Bestand fanden in der Zwischenkriegszeit unter
anderem auch Schriften anderer faschistischer und
nationalsozialistischer Spitzenexponenten wie Benito Mussolini oder
Joseph Goebbels, von Führerfiguren der faschistischen Schweizer
\enquote{Fronten} wie Georges Oltramare, Robert Tobler, Rudolf Henne,
Arthur Fonjallaz und Emil Sonderegger, Rassentheoretikern des 19. und
frühen 20.~Jahrhunderts wie Arthur de Gobineau, Houston Stewart
Chamberlain, Jörg Lanz von Liebenfels, Hans F. K. Günther, Alfred
Ploetz, Erwin Baur, Eugen Fischer, Fritz Lenz und Ernst Rüdin, des
antisemitischen US-Industriellen Henry Ford oder des NS-Chefideologen
Alfred Rosenberg. Das NSDAP-Leitorgan \enquote{Völkischer Beobachter}
ist für die Jahre 1932 bis 1945 vorhanden. Mehrere Periodika der
Schweizer Frontenbewegung besitzt das Sozialarchiv vollständig: Die 1931
bis 1943 erschienenen Parteiorgane der Nationalen Front (\enquote{Der
Eiserne Besen}, \enquote{Die Front}), die \enquote{Nationalen Hefte}
(1934--1945) und die \enquote{Heimatwehr} (1934--1935).

Der Jahresbericht 1932 -- der ungefähr zur Zeit der
nationalsozialistischen Bücherverbrennungen vom 10.~Mai 1933 verfasst
wurde -- hielt zur Sammlungsstrategie im eskalierenden
\enquote{Zeitalter der Extreme} (E. Hobsbawm) folgendes fest:
\enquote{Es handelt sich bei der Auswahl der Anschaffungen weder um die
persönlichen Auffassungen der Besucher, noch um diejenigen der Leitung.
Es handelt sich um mehr, um die Wahrung eines wertvollsten Kulturgutes:
um eine Dokumentierung der Gegenwart an die Zukunft. Eine Aufgabe, an
die man nur mit starkem Verantwortungsbewusstsein wird herantreten
dürfen. Die Anschaffung eines Buches bedeutet also noch längst nicht
dessen Bejahung durch die Bibliotheksleitung {[}...{]}} (Zentralstelle
für Soziale Literatur der Schweiz 1932: 4). Diese Strategie wurde auch
von höchster Stelle anerkannt: Als der Schweizer Bundesrat 1940 ein
Verbot des Vertriebs kommunistischer Literatur erließ, erlaubte eine
Ausnahmebewilligung der Institution, die illegalen Schriften zur
wissenschaftlichen Nutzung weiterhin im Lesesaal zugänglich zu machen
(Häusler 2006: 10). Auch im Kalten Krieg wurde diese Strategie
weiterbetrieben, wobei -- etwa im Unterschied zum Schweizerischen
Ostinstitut -- keine einseitige Position der \enquote{Feindbeobachtung}
eingenommen, sondern weiterhin die umfassende \enquote{Dokumentierung
der Gegenwart an die Zukunft} angestrebt wurde.

\hypertarget{aktuelle-bestandsuxfcbersicht}{%
\section*{Aktuelle
Bestandsübersicht}\label{aktuelle-bestandsuxfcbersicht}}

Heute ist in den einzelnen Abteilungen des Sozialarchivs rechtsextremes
Material unterschiedlich stark vertreten. Die zurzeit aus etwa 800
Organisationsarchiven und Privatnachlässen bestehende Archivabteilung
umfasst die wesentlichen Gewerkschaften und Arbeitnehmerverbände,
politischen und kulturellen Organisationen der Arbeiterbewegung,
sozialen Bewegungen aus Bereichen wie Feminismus, Pazifismus, Ökologie
oder LGBT sowie gemeinnützigen Vereinigungen und Jugendorganisationen
der Schweiz, aber auch Archive kommunistischer und anderer linksextremer
Organisationen sowie Bestände aus dem rechtspopulistischen Spektrum,
etwa den Nachlass des Schweizer Antiimmigrationspioniers der 60er und
70er Jahre James Schwarzenbach, das Plakatarchiv von Alexander Segerts
Werbeagentur \enquote{Goal} (mit Schwerpunkt auf Wahl- und
Abstimmungswerbung der Schweizerischen Volkspartei) oder das
Vereinsarchiv der drogenpolitischen \enquote{Aktion betroffener
Anrainer} der 90er Jahre. Eigentliche rechtsextreme Organisationsarchive
oder Personennachlässe sind keine vorhanden. Der -- dem Sozialarchiv vom
Urheber zu dessen Lebzeiten geschenkte -- Nachlass von James
Schwarzenbach setzt erst nach dem Zweiten Weltkrieg ein und deckt die
Zeit seiner Mitgliedschaft bei der Nationalen Front in den 30er Jahren
nicht ab. Hingegen enthält das Archiv der Sozialdemokratischen Partei
der Schweiz eine zur \enquote{Feindbeobachtung} angelegte, umfangreiche
Sammlung von Flugblättern, Zeitungen, Büchern und teilweise sogar
Mitgliederlisten verschiedener faschistischer Organisationen in der
Schweiz der 30er und 40er Jahre. Einen ähnlichen Charakter hat die
voluminöse Dokumentation zum Schweizer Rechtsextremismus des letzten
Viertels des 20.~Jahrhunderts im Nachlass des Journalisten Jürg
Frischknecht, welche Publikationen verschiedener rechtsextremer und
rechtsesoterischer Gruppierungen und Zirkel sowie Frischknechts
Korrespondenz mit Exponenten dieser Szene enthält. In der umfangreichen
Sammlung von \enquote{Gretlers Panoptikum zur Sozialgeschichte}, die
gegenwärtig vom Schweizerischen Sozialarchiv übernommen wird, finden
sich ebenfalls Dossiers zur Schweizer Frontenbewegung der 30er und 40er
Jahre, dem italienischen Faschismus, deutschen Nationalsozialismus und
deren Schweizer Sympathisanten sowie dem Rechtsextremismus nach 1945,
die teilweise auch Dokumente aus der Provenienz dieser Gruppierungen
selbst enthalten.

Die im Unterschied zu den Provenienzbeständen des Archivs nach dem
Pertinenzprinzip aufgebaute und auf eigener Sammeltätigkeit von analogen
und digitalen Kleinschriften sowie Zeitungsartikeln beruhende
Sachdokumentation enthält eine Vielzahl rechtsextremen Materials. Aus
der Zwischenkriegszeit existieren Dossiers zur schweizerischen
Frontenbewegung mit Programmen, Broschüren, Flugblättern und Exemplaren
der Parteipresse, aber auch gegnerischen Schriften. Hinzu kommen
umfangreiche, ähnlich aufgebaute Dossiers zum deutschen
Nationalsozialismus und italienischen Faschismus. Seit 1960 führt die
Sachdokumentation ein Dossier \enquote{Neonazis, Neue Rechte}, das neben
wissenschaftlichen Broschüren sowie Flugschriften der Antifa-Szene auch
Flugblätter und Programme rechtsextremer Organisationen enthält. In der
seit 2016 aufgebauten Sammlung digitaler Kleinschriften findet sich das
aktuelle Programm der \enquote{Partei national orientierter Schweizer}
(PNOS), deren Webauftritt auf Antrag des Sozialarchivs auch im von der
Schweizerischen Nationalbibliothek betriebenen \enquote{Webarchiv
Schweiz} archiviert wird. Dass diese Sammeltätigkeit von den betroffenen
Organisationen selber kaum als Propagandamöglichkeit betrachtet wird,
zeigt der Umstand, dass in jüngster Zeit Anfragen um Übersendung von
Material teilweise unbeantwortet blieben.

In der Bibliotheksabteilung, die insgesamt etwa 175.000 Monografien,
1.500 laufende Periodika-Titel und umfangreiche historische Zeitungs-
und Zeitschriftenbestände umfasst, findet sich neben den historischen
rassistischen, nationalsozialistischen und faschistischen Beständen auch
neurechte, alt- und neonazistische Literatur der Zeit von 1945 bis in
die Gegenwart. Dazu zählen etwa die Schrift \enquote{Konservative
Evolution: Das Ende des Säkularismus} (1968) des ehemaligen
SS-Obersturmbannführers Franz Riedweg, dem 1944 die Schweizer
Staatsbürgerschaft entzogen worden war, die Autobiografie des britischen
Faschistenführers Oswald Mosley von 1973 oder zwei programmatische
Schriften von Jean-Marie Le Pen von 1985, aus dem Bereich der Periodika
beispielsweise die von der \enquote{Nationale{[}n{]} Basis Schweiz} in
den 70er Jahren publizierte Zeitschrift \enquote{Visier}. Von den
intellektuellen Vordenkern der \enquote{Nouvelle Droite} ist der
Franzose Alain de Benoist mit sechs, der Deutsche Henning Eichberg mit
drei, der Franzose Pierre Krebs mit einem und der Schweizer Armin Mohler
gar mit 14 Titeln vertreten.

Neben diesen politisch-programmatischen und philosophisch-ideologischen
Schriften finden sich auch Titel aus dem Bereich rechtsextremer
Geschichtsfälschung. Die Holocaustleugnung als krasseste Form, die in
der Schweiz per Volksbeschluss seit 1995 unter Strafe steht, ist im
Nebis-Verbund insbesondere im Bestand des (unter anderem auf jüdische
Geschichte und den Zweiten Weltkrieg spezialisierten) Archivs für
Zeitgeschichte der ETH Zürich durch Thies Christophersen und Carlo
Mattogno sowie die Schweizer Gaston-Armand Amaudruz, René-Louis Berclaz,
Bernhard Schaub und Henri Roques vertreten. Im Sozialarchiv finden sich
aus diesem Feld Titel des Schweizers Jürgen Graf und des Amerikaners
Arthur R. Butz. Von David Irving weist der Bestand des Sozialarchivs 13
Bücher aus, das jüngste allerdings von 1987, einem Jahr, bevor Irving
öffentlich den Holocaust zu leugnen begann. Aus dem Bereich der Leugnung
der nazideutschen Schuld am Zweiten Weltkrieg sind zwei Titel von David
L. Hogan vorhanden.

Verschiedene neurechte und rechtsextreme Verlage Deutschlands, deren
Präsenz an den großen Buchmessen in den letzten Jahren zu einem
Dauerpolitikum geworden ist, sind im Bibliotheksbestand des
Sozialarchivs mit wenigen Titeln vertreten. Vom Grabert-Verlag sind dies
David L. Hoggans \enquote{Der erzwungene Krieg} (1977), ein Buch über
Lateinamerika von 1981 sowie Alain de Benoists \enquote{Heide sein zu
einem neuen Anfang} (1982). Von Bublies finden sich Günter Bartschs
Otto-Strasser-Biografie (1990) sowie ein kompletter Satz von
\enquote{Wir selbst -- Zeitschrift für nationale Identität}
(1980--2001). Von Antaios umfasst der Sozialarchiv-Bestand neben dem
bereits erwähnten posthumen Sieferle-Bestseller Karlheinz Weißmanns
Mohler-Biografie (2010) und Mohlers Streitschrift \enquote{Gegen die
Liberalen} (2017) sowie Erik Lehnerts \enquote{Wozu Politik?} (2010),
Manfred Kleine-Hartlages \enquote{Besichtigung des Schlachtfelds} (2016)
und das Traktat \enquote{Mit Linken leben} (2017) von Martin Lichtmesz
und Caroline Sommerfeld. Der Verlag Manuscriptum ist durch einen
weiteren posthumen Titel Sieferles vertreten: \enquote{Das
Migrationsproblem: Über die Unvereinbarkeit von Sozialstaat und
Masseneinwanderung} (2018). Vom Jungeuropa-Verlag führt das Sozialarchiv
den Titel \enquote{Marx von rechts} (2018) von Alain de Benoist und
anderen. Vom Regin-Verlag sind Biografien von zwei
konservativ-autoritären Vordenkern der Zwischenkriegszeit, Arthur
Moeller van den Bruck und Othmar Spann, des neurechten Historikers und
AfD-Funktionärs Sebastian Maaß vorhanden (beide 2010). Vom
Vowinckel-Verlag finden sich die Schriften \enquote{Marxismus? Ein
Aberglaube! Naturwissenschaft widerlegt die geistigen Grundlagen von
Marx und Lenin} (1972) und \enquote{Volksfront droht! Die Verschwörung
der Linken} (1976). Der bereits 1952 von Helmut Sündermann, der in der
Nazi-Zeit stellvertretender Reichspressechef gewesen war, gegründete
Druffel-Verlag ist sogar mit 18 Titeln aus den 70er Jahren vertreten,
die hauptsächlich Themen des Zweiten Weltkriegs aus einer ultrarechten
Perspektive darstellen. Viele Titel der genannten Verlage sind im
Nebis-Verbund ausschließlich beim Sozialarchiv greifbar. Hingegen findet
sich von den wenigen und sehr kleinen Schweizer Verlagen dieses
Spektrums (wie \enquote{Neue Zeitwende} oder \enquote{Editions de
Cassandre}) im Sozialarchiv nichts.

Bei den laufenden Zeitschriften führt das Sozialarchiv aus der Schweiz
einen klar rechtsextremen Titel, den sonst keine einzige Bibliothek des
Nebis-Verbundes verzeichnet: Die Parteizeitschrift der PNOS, die mit
ihrem Titel \enquote{Harus Magazin} an den Führergruß der Schweizer
Frontenbewegung der 30er und 40er Jahre anknüpft (sowie die
Vorgängerzeitschrift \enquote{Zeitgeist}). Hinzu kommt aus der Schweiz
etwa ein halbes Dutzend Zeitschriftentitel aus dem Graubereich zwischen
dem ultrarechten Rand rechtspopulistischer Parteien und dem offenen
Rechtsextremismus. Bei den ausländischen Titeln führt das Sozialarchiv,
wiederum als einzige Bibliothek des Nebis-Verbundes, die deutsche
\enquote{Nationalzeitung} und die \enquote{Junge Freiheit}.

\hypertarget{erwerbungsstrategie-und-benutzungsbedingungen}{%
\section*{Erwerbungsstrategie und
Benutzungsbedingungen}\label{erwerbungsstrategie-und-benutzungsbedingungen}}

Die Präsenz neurechter und neonazistischer Literatur im
Bibliotheksbestand des Sozialarchivs folgt insgesamt der allgemeinen
Erwerbungsstrategie: Angeschafft werden programmatische Titel von einer
gewissen Prominenz oder Bücher, welche Themen in den
Schwerpunktbereichen des Sozialarchivs aus einer neurechten oder
rechtsextremen Perspektive beleuchten. Dabei werden die
Neuerscheinungslisten neurechter und rechtsextremer Verlage nicht
systematisch als Erwerbungsgrundlagen ausgewertet. Die Erwerbung ihrer
Titel erfolgt aufgrund des medialen Echos und Rezensionen außerhalb
neurechter Publikationen und ihrer thematischen Relevanz. Zu betonen ist
auch, dass im Bibliotheksbestand des Sozialarchivs die geschichts-,
sozial- und kulturwissenschaftliche Forschungsliteratur über die
verschiedenen Spielarten des historischen und aktuellen
Rechtsextremismus die Primärtexte quantitativ bei Weitem übertrifft und
von letzteren, wo vorhanden, bevorzugt wissenschaftlich-kritische
Editionen erworben werden (wie bei \enquote{Mein Kampf} von 2016).

Ein Mittel, rechtsextreme und neurechte Literatur den Benutzenden als
solche transparent zu machen, stellt natürlich die Anreicherung mit
Metadaten dar. Bei einer Forschungsinfrastruktureinrichtung sollte hier
insbesondere die Einordnung solcher Literatur im Feld dessen erfolgen,
was je nach wissenschaftlicher Disziplin als \enquote{Quellen},
\enquote{Primärtexte} oder \enquote{Forschungsdaten} bezeichnet wird.
Der von juristischer Seite vorgebrachte Einwand, eine besondere
Kennzeichnung von Literatur mit strafbarem Inhalt im Katalog sei zu
unterlassen, \enquote{weil dies nur die Benützer neugierig macht}
(Künzle 1992: 248), greift in diesem Zusammenhang, bei dem nicht die
strafrechtliche Relevanz des Inhalts, sondern seine politische Verortung
im Zentrum steht, kaum.

Bei den einschlägigen Beständen des Sozialarchivs zeigt sich
diesbezüglich ein durchzogenes Bild. Die Materialien in der thematisch
gegliederten und in einer vom Bibliothekskatalog separaten
online-Datenbank (www.sachdokumentation.ch) nachgewiesenen
Sachdokumentation sind durch entsprechende Einordnung in die Systematik
unmissverständlich gekennzeichnet. Anders sieht es teilweise bei den
Metadaten zum Bibliotheksbestand im Nebis-Katalog aus, der auch die
Daten historisch gewachsener Zettelkataloge des analogen Zeitalters
integriert hat und wo die Beschlagwortung durch die verschiedenen
Verbundbibliotheken sowie Fremddatenübernahmen sehr viel heterogener
(und bei Altbeständen lückenhaft) ist. Die wenigen Monografien des
Sozialarchivs aus dem Bereich der Holocaustleugnung sind mit
Schlagwörtern wie \enquote{Holocaust denial literature} oder
\enquote{Geschichtsfälschungen~+~Geschichtsrevisionismus}
gekennzeichnet. Streng genommen ist diese Metadatierung falsch, da es
sich dabei ja nicht um Literatur \emph{über} Holocaustleugnung und
Geschichtsfälschung handelt, sondern um selber in diesem Feld zu
verortende Schriften. Bei anderen rechtsextremen Titeln fehlen Metadaten
oder beziehen sich die Schlagwörter \enquote{neutral} auf den Sachinhalt
ohne politische Verortung. Bei den Periodika wird das \enquote{Harus
Magazin} unter anderem mit den Schlagwörtern \enquote{Rechtsradikale
Partei}, \enquote{Nationalismus} und \enquote{Rechtspartei} verortet,
absonderlicherweise aber auch mit dem vermutlich aus dem PNOS-eigenen
Vokabular stammenden Begriff \enquote{Volksstaat}. Die
\enquote{Nationalzeitung} verfügt im Katalogeintrag über keine
Schlagworte, die \enquote{Junge Freiheit} wird mit \enquote{Neue Rechte}
politisch verortet.

Für die Benutzung rechtsextremen Materials kennt das Sozialarchiv im
Allgemeinen kaum besondere Restriktionen. Während Archiv- und
Dokumentationsbestände generell nur im Lesesaal konsultiert werden
können und bei der Bestellung -- allerdings aus rein statistischen
Gründen -- der Benutzungszweck erfragt wird, gilt die
Lesesaalbeschränkung bei den Bibliotheksbeständen lediglich für Titel,
die aus konservatorischen Gründen oder aufgrund ihrer Seltenheit nicht
in die Heimausleihe gegeben werden können. Eine Spezialregelung besteht
diesbezüglich lediglich im Hinblick auf strafrechtlich relevante Texte
aus dem Bereich der Holocaustleugnung, deren Einsichtnahme auf den
Lesesaal beschränkt ist. Diese Praxis folgt dem im bundesdeutschen
Strafrecht als \enquote{Sozialadäquanz} bekannten Grundsatz, der die
eingeschränkte Benutzung von Literatur mit strafrechtlich relevantem
Inhalt bei berechtigtem Interesse, etwa zur wissenschaftlichen
Auswertung oder im Dienste der \enquote{staatsbürgerlichen Aufklärung},
zulässt (Künzle 1992: 249). In der Öffentlichkeitsarbeit
(Veranstaltungen, Newsletter, et cetera) wurde das Vorhandensein von und
der Umgang mit rechtextremem Material im Sozialarchiv bislang nur
marginal thematisiert.

Wie ist diese jahrzehntelange Praxis hinsichtlich der Frage zu bewerten,
ob in einem demokratischen Staat öffentlich finanzierte Bibliotheken
rechtsextreme und andere demokratiefeindliche Literatur
\enquote{neutral} behandeln sollen oder diese aber wegen politischer
Kontaminationsgefahr aus den Beständen fernzuhalten ist? Zunächst ist
auf eine begriffliche Differenzierung hinzuweisen: Wie beim
völkerrechtlichen Neutralitätsbegriff die außenpolitische Neutralität
eines Staates nicht mit dessen Gesinnungsneutralität zusammenfällt, so
bedeutet eine \enquote{Erwerbungsneutralität} bei der Akquisition
politischen Materials keineswegs, dass die entsprechende Institution
eine Position politischer Neutralität zwischen demokratischen und
demokratiefeindlichen Kräften einnehmen würde.

Die Gefahr, dass durch die Erwerbung ausgewählter rechtsextremer Titel
ein Beitrag zur Verbreitung dieses Gedankenguts geleistet wird, ist bei
einer wissenschaftlichen Spezialbibliothek (im Unterschied zu
öffentlichen allgemeinen Bibliotheken) als gering einzuschätzen. Erstens
kann die Fachkompetenz und politische Reife der überwiegenden Mehrzahl
der Benutzenden als hinreichend groß veranschlagt werden, dass sie ihre
Ansichten zum Rechtsextremismus nicht durch das Vorhandensein einiger
Titel mit entsprechender Ausrichtung beeinflussen lassen würden.
Zweitens ist das Wegschließen rechtsextremer Literatur im
bibliothekarischen Giftschrank im digitalen Zeitalter weitgehend
wirkungslos geworden. Wer entsprechendes Material lesen möchte, kann
sich dieses online problemlos beschaffen. \enquote{Mein Kampf} und die
\enquote{Protokolle der Weisen von Zion} sind beispielsweise schon seit
mindestens zwei Jahrzehnten auf rechtsextremen Webseiten im Volltext
zugänglich -- eingebettet in ein entsprechendes Setting. Drittens dürfte
die propagandistische Wirkung von rechtsextremer Seite erhobener
Zensur-Vorwürfe größer sein als die eindämmende Funktion durch den
Ausschluss solcher Titel aus der Erwerbung und Benutzung.

Gelten diese Bemerkungen für offen als rechtsextrem erkennbare
Literatur, so ist die Situation bei subtiler argumentierendem
\enquote{neurechtem} Schrifttum komplizierter. Die intellektuellen
Vordenker dieser Strömung propagieren seit geraumer Zeit einen
Kulturkampf, der sich an eine in der Zwischenkriegszeit vom
italienischen Kommunisten Antonio Gramsci entwickelte Strategie anlehnt.
Dabei geht es darum, als Vorbereitung einer Machtübernahme zunächst den
vorpolitischen, kulturellen Raum zu besetzen, ideologische Inhalte in
die gesellschaftliche Diskussion zu bringen, Akzeptanz für sie zu
schaffen und den demokratischen Kräften die \enquote{Diskurshoheit}
streitig zu machen. Dazu wird bewusst Distanz zu allgemein bekannten und
diskreditierten alt- und neonazistischen Positionen gewahrt. So treten
an die Stelle der biologistischen Rassentheorien Ideologeme, die von der
Forschung als \enquote{Rassismus ohne Rassen} charakterisiert werden und
unter beschönigenden Bezeichnungen wie \enquote{Ethno-Pluralismus} die
räumliche Trennung angeblich nicht vereinbarer \enquote{Kulturen}
postulieren (Koller 2009: 89--97). Und statt der plumpen und illegalen
Holocaustleugnung wird auf subtilere Weise versucht,
nationalsozialistische und andere rechtsextreme Verbrechen zu
relativieren und zu verharmlosen und deren Ausmaß als angeblichen
Gegenstand wissenschaftlicher Kontroversen erscheinen zu lassen. Die
korrekte Einordnung solcher Schriften setzt beim zuständigen Fachreferat
erhebliche Kenntnisse voraus. Unter dieser Perspektive kritisch zu
beurteilen sind aus dem Bestand des Sozialarchivs ein bis zwei Dutzend
unzureichend beschlagwortete Bücher zu Themen des Zweiten Weltkriegs aus
den 70er und 80er Jahren. In jüngerer Zeit scheinen keine solchen Titel
mehr erworben worden zu sein.

\hypertarget{fazit}{%
\section*{Fazit}\label{fazit}}

Der auf einer jahrzehntelangen Praxis beruhende Umgang des
Schweizerischen Sozialarchivs als einer wissenschaftlichen, aber auch
zwischen Wissenschaft und breiter Öffentlichkeit vermittelnden
Institution mit rechtsextremem Material beruht insgesamt auf der
Überzeugung, dass die Frage der Sammlung und Vermittlung beziehungsweise
Nichtsammlung solchen Materials nicht an der Frage der
\enquote{politischen Neutralität} der Bibliothek festgemacht werden
kann. Die Nichterwerbung solchen Materials käme für eine
wissenschaftliche Spezialbibliothek mit den Themenschwerpunkten des
Sozialarchivs einer teilweisen Missachtung ihres Sammelauftrags gleich
und allzu restriktive Benutzungsbedingungen wären weder zweckmäßig noch
der Kompetenz der Benutzenden angemessen. Bibliotheken als demokratische
Institutionen sollten antidemokratischer Literatur mit den Waffen der
Demokratie begegnen, also nicht mit Zensur und Verschweigen, sondern mit
Transparenz und Kritik.

\hypertarget{bibliografie}{%
\section*{Bibliografie}\label{bibliografie}}

Häusler, Jacqueline (2006): 100 Jahre soziales Wissen: Schweizerisches
Sozialarchiv 1906--2006. Zürich: Schweizerisches Sozialarchiv.

Koller, Christian (2009): Rassismus (UTB Profile). Paderborn: Ferdinand
Schöningh.

Koller, Christian (2015): Bibliotheksgeschichte als \emph{histoire
croisée}: Das Schweizerische Sozialarchiv und das Phänomen des Exils,
in: Ball, Rafael und Stefan Wiederkehr (Hg.): Vernetztes Wissen. Online.
Die Bibliothek als Managementaufgabe: Festschrift für Wolfram Neubauer
zum 65.~Geburtstag. Berlin: De Gruyter. S. 365--392.

Künzle, Hans Rainer (1992): Schweizerisches Bibliotheks- und
Dokumentationsrecht: Das Recht der Bibliotheken, Archive, Museen und
Dokumentationsstellen in der Schweiz mit rechtsvergleichenden Hinweisen
auf das deutsche, französische, englische und amerikanische Recht.
Zürich: Schulthess Polygraphischer Verlag.

Schweizerisches Sozialarchiv (2018): Leitbild vom 11.4.2018: URL:
\url{https://www.sozialarchiv.ch/wp-content/uploads/fileadmin/user_upload/Sozialarchiv/Dokumente/PDFs/Sozialarchiv/leitbild.pdf}

Zentralstelle für Soziale Literatur der Schweiz (1932): Jahresbericht
1932.

%autor
\begin{center}\rule{0.5\linewidth}{\linethickness}\end{center}

\textbf{Christian Koller} hat Geschichte, Betriebswirtschaftslehre und
Politikwissenschaft studiert und ist Direktor des Schweizerischen
Sozialarchivs, Titularprofessor für Geschichte der Neuzeit an der
Universität Zürich sowie Mitglied der Kommission der Schweizerischen
Nationalbibliothek. \url{http://orcid.org/0000-0001-9701-0122}

\end{document}
